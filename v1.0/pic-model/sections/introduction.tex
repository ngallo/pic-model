This model is derived from a natural observation of the delegation pattern and the multi-hop nature of distributed systems. Even in a paper-based artifact format, delegation involves the delegator (who signs) and the delegate (who proves their identity). In such a model, an artifact loses its meaning if the delegate is removed; anyone who acquires it can claim its use, thereby nullifying the intrinsic value of the delegation artifact itself. The two minimal required entities are the delegator and the delegate. Therefore, the validity of both the delegator and the delegate must be proven---two single inputs. Any other method that introduces complexity only increases the attack surface, rendering the approach weaker from a security perspective.

Naturally, artifact-centric security models are widely used today, and it may seem counter-intuitive that they are inadequate for securing distributed systems. The problem, as anticipated, stems from the artifact's inherent coupling with the delegator (the artifact holder). Current delegation models operate only under severe restrictions:
\begin{itemize}
    \item The initial single hop requires authenticated transport channels (e.g., TLS authentication), essentially binding the delegate's identity to the transport channel.
    \item Subsequent calls over authenticated transport channels are often maintained via token exchange mechanisms.
    \item Second and subsequent hops are often assumed to be protected solely by network isolation (VPNs, firewalls, private networks), leading to situations where identity continuity cannot be reliably preserved.
\end{itemize}

For example, in systems like Apache Kafka, \textbf{relying on a traditional static security artifact is often impractical for safe and functional transmission}, as this immediately creates a large attack vector. If the signature is removed, it ceases to be a security artifact; if it is encrypted, there is no guarantee it will be processed before expiration. This fragmented approach necessitates multiple methods to handle exceptions, ultimately leading to superior complexity and an increased number of vulnerabilities.

To fully grasp the paradigm shift, we \textbf{MUST} first formalize the inherent multi-hop nature of distributed systems.