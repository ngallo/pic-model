\documentclass[11pt]{article}
\usepackage{amsmath,amssymb,amsthm}
\usepackage{hyperref}

\renewcommand{\qedsymbol}{}

\title{Authority is a Continuous System}
\author{Nicola Gallo}
\date{8 December 2025}

\begin{document}
\maketitle

\section{Motivation}

In prior work~[1], we formally showed that Proof-of-Possession (PoP)--based
authorization models admit confused deputy conditions.
Because authority is exercised solely through possession of a
transferable artifact, deputies may reinterpret privileges outside the
causal intent of the originator.

The same work showed that the Provenance Identity Continuity (PIC) model
eliminates the confused deputy entirely.
By enforcing monotonic authority restriction and origin-bounded
execution, PIC makes the confused deputy condition non-formulable:
no privilege can be exercised unless it was present at the execution
origin and causally propagated through a verifiable chain.

This leads to a fundamental insight: authority is not a transferable
object, but a continuity property of execution.
Once this is understood---against decades of research in access
control---it becomes clear that many existing models have incorrectly
treated authority as a form of possession.

This paper elevates the PIC model to a more abstract and general form,
analyzing authority as a continuous system and formalizing its
implications for trust, governance, concurrency, and zero-trust
architectures.

\section{Authority Continuity Principle}

Let $P$ be a finite set of principals, $O$ a set of operations, and $R$ a
set of resources.
A privilege is a pair $(o,r) \in O \times R$.

Let $p_0 \in P$ be the immutable originating principal of an execution.
The set $\mathrm{ops}_0 \subseteq O \times R$ denotes the authority
explicitly exercised within this execution.
Importantly, $\mathrm{ops}_0$ is a subset of the total authority of $p_0$;
only this subset participates in execution continuity.

In prior work~[1], execution was intentionally modeled as:
\[
\pi = \langle (p_0, \mathrm{ops}_0), (p_1, \mathrm{ops}_1), \dots,
(p_n, \mathrm{ops}_n) \rangle,
\quad
\mathrm{ops}_{i+1} \subseteq \mathrm{ops}_i,
\]
in order to demonstrate the elimination of the confused deputy under
monotonic restriction.

For a more general and abstract treatment, we model execution as a
continuous flow of authority rather than a reassignment between
principals.

\begin{quote}
\emph{The originating principal of an execution is immutable.}
\end{quote}

Execution is represented as a chain of authority states:
\[
\pi = \langle \alpha_0, \alpha_1, \dots, \alpha_n \rangle,
\quad
\alpha_i = (p_0, \mathrm{ops}_i, e_{i-1}, e_i),
\]
where $e_i \in P$ is the executor performing hop $i$.

Monotonicity holds for all $i$:
\[
\mathrm{ops}_{i+1} \subseteq \mathrm{ops}_i.
\]

\section{Guardrails}

Authority continuity is enforced by invariant constraints evaluated at
every execution step $i$.
Guardrails validate continuity; they never grant authority.

\subsection{Origin Guardrail}
\[
\forall i:\; p_0^{(i)} = p_0^{(0)}.
\]

\subsection{Operational Guardrail}
\[
\forall i:\; \mathrm{ops}_{i+1} \subseteq \mathrm{ops}_i.
\]

\subsection{Executor Guardrail}
\[
\forall i:\; \mathrm{ExecCont}(p_0, e_i, e_{i+1}, \mathrm{ops}_{i+1}).
\]

\subsection{Temporal Guardrail}

Temporal validity is defined by a predicate, not by simple set
containment.

Let $\tau_i$ be the current time, included in the execution context.
Let $T_i$ denote a temporal constraint associated with state $\alpha_i$.

\[
\mathrm{TimeValid}(T_i, \tau_i) \in \{\mathsf{true}, \mathsf{false}\}.
\]

The temporal constraint $T_i$ may represent:
\begin{itemize}
\item an absolute time range $[t_{\min}, t_{\max}]$ such that
      $\mathrm{TimeValid}(T_i, \tau_i)$ holds iff
      $\tau_i \in [t_{\min}, t_{\max}]$;
\item a relative duration from an origin time $t_0$
      (e.g.\ valid for $\Delta t$ after $t_0$), such that validity holds
      iff $\tau_i - t_0 \leq \Delta t$;
\item an unbounded (infinite) interval, in which case
      $\mathrm{TimeValid}(T_i, \tau_i)$ is always $\mathsf{true}$.
\end{itemize}

If no temporal constraint is specified, execution is temporally
unbounded.

\subsection{Contextual Guardrail}

Context is validated by predicate, not merely by set inclusion.

Let $C_i$ be the execution context associated with $\alpha_i$ and
$X_i$ external signals (including time and environmental conditions).

\[
\mathrm{CtxValid}(C_i, X_i) \in \{\mathsf{true}, \mathsf{false}\}.
\]

Simple containment $C_{i+1} \subseteq C_i$ is a sufficient but not
necessary condition.
Context validity may depend on policies, environment, or other external
constraints.

\section{Executor Continuity Predicate}

\begin{definition}[Executor Continuity]
The predicate
\[
\mathrm{ExecCont}(p_0, e_i, e_{i+1}, \mathrm{ops})
\]
holds iff executor $e_{i+1}$ is a causally valid continuation of
execution from $e_i$ under origin $p_0$ and authority $\mathrm{ops}$.
\end{definition}

Executor continuity may depend on implementation-specific predicates,
including but not limited to:
\begin{itemize}
\item proof that $e_{i+1}$ is the executor it claims to be,
\item proof of required execution attributes or environment,
\item proof of contextual binding (e.g.\ request provenance),
\item proof that execution proceeds within authority $\mathrm{ops}$.
\end{itemize}

Crucially, $\mathrm{ExecCont}$:
\begin{itemize}
\item does not grant authority,
\item does not expand authority,
\item does not preserve authority,
\item does not substitute authority continuity.
\end{itemize}

It merely validates that execution at hop $i+1$ is causally derived from
execution at hop $i$ and is therefore eligible to participate in
authority continuity.

\section{Authority Succession}

Authority evolves by applying all guardrails atomically.

We enrich authority states as:
\[
\alpha_i = (p_0, \mathrm{ops}_i, e_{i-1}, e_i, T_i, C_i).
\]

Given proposed inputs
\[
(e_{i+1}, \mathrm{ops}_{i+1}, T_{i+1}, C_{i+1}),
\]
we construct the candidate successor:
\[
\alpha_{i+1} =
(p_0, \mathrm{ops}_{i+1}, e_i, e_{i+1}, T_{i+1}, C_{i+1}).
\]

We define the combined guardrail predicate:
\[
\begin{aligned}
\mathrm{Guard}(\alpha_i, \alpha_{i+1}) \equiv\;&
\big(p_0^{(i+1)} = p_0^{(i)}\big)
\land \big(\mathrm{ops}_{i+1} \subseteq \mathrm{ops}_i\big) \land{}\\
& \mathrm{ExecCont}\big(p_0, e_i, e_{i+1}, \mathrm{ops}_{i+1}\big) \land{}\\
& \mathrm{TimeValid}(T_{i+1}, \tau_{i+1})
\land \mathrm{CtxValid}(C_{i+1}, X_{i+1}) .
\end{aligned}
\]

The succession function is:
\[
\mathsf{Next}(
  \alpha_i,\,
  e_{i+1},\,
  \mathrm{ops}_{i+1},\,
  T_{i+1},\,
  C_{i+1}
)
=
\begin{cases}
\alpha_{i+1} &
\text{if } \mathrm{Guard}(\alpha_i, \alpha_{i+1}) = \mathsf{true},\\[4pt]
\mathsf{halt} & \text{otherwise.}
\end{cases}
\]

Succession:
\begin{itemize}
\item constructs a candidate $\alpha_{i+1}$,
\item applies all guardrails as a single predicate,
\item returns $\alpha_{i+1}$ only if continuity is preserved,
\item otherwise halts execution at $\alpha_i$.
\end{itemize}

Succession preserves continuity and never introduces new authority.

\section{Transactional Structure of Authority Continuity}

Authority continuity admits both sequential and parallel evolution.

\subsection{Sequential Continuation}

Repeated application of $\mathsf{Next}$ yields linear chains:
\[
\alpha_0 \rightarrow \alpha_1 \rightarrow \dots \rightarrow \alpha_n.
\]
Failure of a single transition halts execution.

Sequential composition is purely structural: the existence of a syntactic
sequence $\alpha_i \rightarrow \alpha_{i+1}$ does not, by itself,
guarantee that execution is authorized to proceed.
Each transition in the chain is realized only when the corresponding
call to $\mathsf{Next}$ returns $\alpha_{i+1}$ (i.e.\ when
$\mathrm{Guard}$ holds and any governance decisions allow it); otherwise
execution halts at $\alpha_i$.

\subsection{Parallel Continuation (Fork)}

Given a state $\alpha_i$, multiple successors may be generated:
\[
\alpha_{i+1}^{(1)}, \dots, \alpha_{i+1}^{(k)},
\]
each constructed via $\mathsf{Next}$ with different parameters.

Every branch must independently satisfy all guardrails.
Forking does not duplicate or expand authority: each branch maintains
monotonicity relative to its own execution line.

The existence of multiple successor candidates does not, by itself,
authorize execution along any branch.
For each branch $j$, the transition
$\alpha_i \rightarrow \alpha_{i+1}^{(j)}$ becomes effective only if the
corresponding $\mathsf{Next}$ call returns $\alpha_{i+1}^{(j)}$ and any
governance decisions permit that branch to proceed; otherwise the branch
halts at $\alpha_i$.

\subsection{Join (Convergent Continuation)}

Let $\alpha_i^{(1)}, \dots, \alpha_i^{(k)}$ be states that share the
same origin $p_0$.
A join function produces a new candidate state:
\[
\mathsf{Join}(\alpha_i^{(1)}, \dots, \alpha_i^{(k)}) \rightarrow \alpha'_i.
\]

Let:
\[
\mathrm{ops}' \subseteq \bigcap_j \mathrm{ops}_i^{(j)}, \quad
T' \subseteq \bigcap_j T_i^{(j)}, \quad
C' \subseteq \bigcap_j C_i^{(j)}.
\]

Join is thus a transformation that produces $\alpha'_i$ with authority,
time, and context no greater than what is already present in the input
branches.
Join by itself does not imply authorization to execute.
The resulting state $\alpha'_i$ must be treated as a candidate and,
for any subsequent step
$\alpha'_i \rightarrow \alpha'_{i+1}$, must again be passed through:
\begin{itemize}
\item continuity guardrails via $\mathsf{Next}$ and $\mathrm{Guard}$,
\item any applicable governance decisions via $\mathrm{PDP}$.
\end{itemize}

Authority after join can never exceed the authority available before
join.

\section{Trust, Possession, and Execution}

Authority continuity cleanly separates execution, trust, and possession.

\subsection{Executor Verification}

At each hop exactly one executor performs the concrete action.
The executor is the only element that must demonstrate anything: that it
is in fact the executor it claims to be.

This demonstration binds execution to reality (identity, environment,
attributes) but does not confer additional authority and does not alter
authority continuity.

\subsection{Possession as Binding, Not Authority}

Proof of Possession (PoP) has a purely binding role:
\begin{itemize}
\item it associates identity, attributes, environment, or context with
      execution,
\item it provides \emph{untrusted evidence} that may be evaluated by
      $\mathrm{ExecCont}$, $\mathrm{TimeValid}$, or $\mathrm{CtxValid}$,
\end{itemize}
but it does \emph{not}:
\begin{itemize}
\item provide trusted evidence for continuity, temporal, or contextual
      validation,
\item create authority,
\item preserve authority,
\item propagate authority,
\item justify authority expansion.
\end{itemize}

Possession is orthogonal to continuity.

\subsection{Trust as Continuity}

Trust does not arise from possession of an artifact.
Trust emerges from verified causal continuity.

An executor is considered trustworthy if and only if its execution
constitutes a valid continuation from the immutable origin $p_0$ under
all guardrails and any applicable governance decisions.

Trust may decrease or be revoked over time by changing policies, but such
revocation does not introduce discontinuity: it halts or restricts the
evolution of authority.

\section{Governance as Policy-Constrained Continuity}

Continuity defines what is structurally possible.
Governance defines what is currently permitted.

Let $\mathrm{Policy}$ be the set of governance rules (which may change
over time) and $X_i$ the external context (network state, risk signals,
operational conditions, etc.).

We define the policy decision predicate:
\[
\mathrm{PDP}(\alpha_i, \alpha_{i+1}, X_i, \mathrm{Policy})
\in \{\mathsf{allow}, \mathsf{deny}\}.
\]

A transition $\alpha_i \rightarrow \alpha_{i+1}$ may occur if and only
if:
\[
\mathrm{Guard}(\alpha_i, \alpha_{i+1})
\;\land\;
\mathrm{PDP}(\alpha_i, \alpha_{i+1}, X_i, \mathrm{Policy})
= \mathsf{allow}.
\]

Governance:
\begin{itemize}
\item may deny transitions that are structurally valid,
\item may further restrict operations, time, and context,
\item may terminate execution in response to external conditions,
\end{itemize}
but it cannot:
\begin{itemize}
\item introduce new authority,
\item expand $\mathrm{ops}$ beyond what continuity allows,
\item bypass guardrails.
\end{itemize}

In continuous-authority systems, governance is therefore an inherently
authority-reducing layer.

\section{Zero-Trust Incompatibility of PoP-Based Propagation}

\begin{definition}[Zero-Trust Compatibility]
A system is compatible with the zero-trust model if every execution
advance is produced exclusively by the succession function
$\mathsf{Next}$ and satisfies all guardrails, possibly subject to
governance decisions.
\end{definition}

\begin{definition}[PoP-Based Propagation]
A system is \emph{PoP-based} if there exists an artifact $a$ such that,
for some contexts, authority at a subsequent hop can be
(re)materialized solely on the basis of possession of $a$, independently
of the current authority state.
\end{definition}

\begin{theorem}
No authorization system based on Proof of Possession is compatible with
the zero-trust model under the authority continuity assumption.
\end{theorem}

\begin{proof}
PoP-based propagation allows authority to be reconstructed in contexts
that do not causally derive from the current state $\alpha_i$.
Authority continuity instead requires monotonic, causally derived
succession constrained by $\mathrm{Guard}$ and, optionally, by
$\mathrm{PDP}$.

Therefore, there exist execution steps enabled solely by possession of
$a$ that cannot be represented as $\mathsf{Next}(\alpha_i, \dots)$ with
$\mathrm{Guard} = \mathsf{true}$.
Such steps violate zero-trust compatibility as defined above.
\end{proof}

\begin{quote}
\emph{Systems that derive authority from Proof of Possession are not
zero-trust systems by construction.}
\end{quote}

\section{Conclusion}

Authority is not an artifact, a credential, or a possession.
It is a continuous, causally constrained property of execution rooted in
an immutable origin.

By modeling authority as continuity:
\begin{itemize}
\item authority propagation becomes intrinsically monotonic,
\item concurrency is handled via fork and join under continuity,
\item governance is expressed as a policy layer that only restricts,
\item the zero-trust model emerges as a native property of the system.
\end{itemize}

Systems that treat authority as possession cannot, by construction,
achieve these properties.
Authority continuity instead provides a unified foundation for
re-thinking security, identity, and access control in distributed
systems.

\section*{Acknowledgments}

The author used automated language assistance tools exclusively for
grammar and phrasing.
All conceptual contributions and formalizations are solely the author’s.

\section*{References}

\begin{enumerate}
\item N.~Gallo.
\emph{Authority Propagation Models: PoP vs PoC and the Confused Deputy Problem.}
Zenodo, 2025.

\item N.~Gallo.
\emph{PIC Model---Provenance Identity Continuity for Distributed Execution Systems.}
Zenodo, 2025.

\item N.~Hardy.
\emph{The Confused Deputy.}
Operating Systems Review, 1988.

\item J.~Kindervag et al.
\emph{Zero Trust Architecture.}
NIST Special Publication 800-207, 2020.

\item Google.
\emph{BeyondCorp: A New Approach to Enterprise Security.}
;login:, 2014.

\item B.~Lampson.
\emph{Protection.}
ACM Operating Systems Review, 1974.
\end{enumerate}

\end{document}
